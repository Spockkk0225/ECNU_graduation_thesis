\vspace{-2.5cm}
\chapter*{\xiaosan \heiti{摘~~~~要}}
\addcontentsline{toc}{chapter}{摘要}
\hspace{-0.5cm}

核密度估计法(Kernel Density Estimation, KDE)已成为一种常见的数据可视化手段,已被广泛地应用于金融风险预测、犯罪聚集分析及交通监控等多个领域中。核密度估计法能够将离散的数据点拟合成连续的分布,实现对任意位置的值预测,并可在后续通过归一化映射到有限范围内实现着色并可视化。

然而大多数现有工作仅考虑空间数据,并使用多维空间上的欧式距离作为距离度量。在本论文中,我们提出了一种新的计算模型,称为时间网络核密度估计(Kemporal Network Kernel Density Estimation,TN-KDE),它将传统的核密度估计法扩展到带有时间数据的路网图中。在时间网络核密度估计中,每个数据点包含坐标信息和时间信息,并且使用最短路径距离作为距离度量,更加符合路网上的实际情况。

为了解决这一问题,本论文设计了一种全新的算法:区间森林法(Range Forest Solution,RFS),用于高效地在时空路网图中计算核密度值。为了进一步支持区间森林法的插入操作,本论文还设计了一种动态算法:动态区间森林法(Dynamic Range Forest Solution,DRFS)。此外,这两种算法还包括一个名为线段点共享(Lixel Sharing,LS)的优化技术,可在相邻的若干线段点中共享大量相似的计算。

最后,针对该算法的计算瓶颈,即大量重复的最短路径计算,本论文还探讨了分布式计算最短路径的可能,并给出了多个常见的分布式计算平台上的最短路径算法实现。


\sihao{\heiti{关键词:}} \xiaosi{核密度估计法,数据可视化,时空数据,最短路径}