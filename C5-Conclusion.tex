\chapter{总结}

	在本文中,我们深度探讨了如何将核密度估计法应用于时空数据路网图上,并提出了一个高效的解决方案,即区间森林法(RFS)。RFS算法通过采用基于内存共享的树形索引结构来同时维护数据点的时间信息和位置信息。这种设计使得它能够在高效处理多个时间窗口查询的同时,避免增加额外的内存开销。此外,通过树形结构组织和管理数据点,查询操作更加迅速,特别是在查询特定时间窗口内的核密度时,只需遍历相关的树节点即可获取所需的信息,大大提高了查询效率。

	为了进一步提升RFS算法的灵活性和实用性,我们开发了动态区间森林法(DRFS)。DRFS算法扩展了原始的RFS算法,通过引入动态结构以支持插入操作,这意味着系统可以在不影响现有查询性能的情况下动态添加新的数据点。更重要的是,DRFS为用户提供了不同等级的量化参数,允许根据实际需求调整计算精度。这种灵活的设计允许用户根据应用场景的具体要求,在计算精度与计算时间之间找到最佳平衡点。例如,在需要快速响应的场景下,可以选择较低的精度以加快计算速度;而在对结果精确度有较高要求的场合,则可以适当提高量化级别。

	除此之外,我们注意到相邻查询点之间的核密度值通常是平滑变化的,并且共享了大量的相似计算步骤。针对这一现象,我们设计了一种称为线段点共享(LS)的技术。LS技术允许将相似的计算结果应用于多个查询点,从而减少不必要的重复计算,显著提升了整体计算效率。通过最大化利用已有的计算资源,LS技术有效降低了整个系统的计算负担,使得大规模数据集的处理变得更加可行。

	并且,我们的解决方案中的核函数是可以替换的,这意味着可以根据具体的分析目标选择最合适的核函数类型。除了传统的线性核函数外,我们的框架还支持更复杂的核函数,包括指数函数和余弦函数等。这为用户提供了一个更加灵活的选择空间,更好地适应不同的应用场景需求,确保结果的准确性和可靠性。
	
	我们还设计了全面的实验流程来评测我们提出的算法的效率。性能实验包括在不同的带宽范围、查询次数、线段点精度、时间窗口大小参数下对比我们提出的算法和基线算法,有效性实验则是在不同的量化等级下对比RFS算法和DRFS算法的差距。	实验结果表明,RFS算法相比于目前最优的算法和基线算法分别有6倍和89倍的性能提升,并且只增加了30\%的额外内存开销;DRFS算法则可以进一步减少40\%的时间开销和60\%的内存开销,且核密度误差不超过5\%。

	考虑到传统的单机计算模式在处理大规模图数据时面临着显著的性能瓶颈,我们还将最短路径算法扩展到了分布式平台上,深入探索了这些算法的分布式计算潜力。分布式计算有两大优点:存储分布式和计算分布式。存储分布式可以将庞大的数据分散存储到不同机器上,避免由于数据过大而无法全部加载到内存的情况;计算分布式则可以将密集的计算任务分散到不同机器上,实现并行计算,提高计算效率。

	由于目前还没有针对分布式图计算的全面总结和技术分类,我们先介绍了分布式图计算的相关背景,包括将图算法迁移到分布式环境中所需要的划分算法和计算模式。划分算法主要研究如何将完整的图切分成若干子图,存放到不同的机器上,并且减少机器之间的通讯所带来的开销;计算模式则是研究如何将计算任务分散到点、边、块或子图上,以实现更好的并行化。
	
	针对这些不同的分布式部署特点,我们选取了六个最常用的且具有代表性的分布式计算平台,包括GraphX、PowerGraph、Flash、Grape、Pregel+和Ligra,这些平台涵盖了各种划分算法和计算模式,有着截然不同的计算特点。我们在这些平台上各自实现了最短路径算法,并在单机和多机环境下进行全面的多线程测试。所有分布式算法均展现了高效的计算流程,这给传统算法的并行化和提供了极大的便利,显示出巨大的应用潜力。
	
	此外,考虑到分布式平台部署的时间成本较高,我们还根据实际的运行结果和部署经验,给出了分布式平台的选用指南。GraphX作为Spark平台的扩展,能够较好的兼容已有的大数据,但基于Hadoop的存储方式和本身架构的局限性使其运行效率较低,相比之下Pregel+作为Pregel的扩展,不仅代码可以兼容,且运行效率也较高;如果对性能较为敏感,则可以选择Grape和PowerGraph,这两个平台都对计算效率做了大量的优化;如果对算法需求较高,则可以选择Flash,Flash提供了大量的算法,不仅包含传统的实现方式,还更新了许多最新论文中提出的优化算法。
	
	最后,本论文依然存在一些不足之处。一方面,即使经过了大量的优化与加速,核密度的计算复杂度依然较高,距离实际应用在毫秒级别的导航类软件上还有很大的差距,并且额外的内存开销在边缘设备这类对资源极度敏感的场景下也是一个不可接受的负担;另一方面,我们发现分布式图计算的横向扩展性较差,多机环境下的表现较差,甚至可能会出现负提升。

	针对这些问题,我们也提出了本工作在后续的改进路线。针对于核密度计算代价较为高昂的问题,我们将以DRFS算法为基础,进一步深入的研究近似算法,且以工业场景落地为目标导向,以用户实际体感为主要指标,淡化理论的近似比数值,同时在时间和内存两方面进行优化,针对分布式场景的横向扩展性较差的问题,需要进一步定位问题来源,可能是由于计算资源已经达到了算法并发度的瓶颈,那么就需要重新设计算法流程,甚至允许增加一部分计算开销来换取更高的并发度,也可能是多机环境本身通讯开销较大,可以选择换取更高效的通讯协议和数据传输格式,或是采用异步架构来减少通讯阻塞带来的损失。