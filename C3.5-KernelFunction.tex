\section{非多项式核函数}
\label{sec7:kernel}

上述的核密度计算中我们使用最为常见的三角核方程作为示例,由于其可被划分为查询向量$\mathbf{Q}$和聚合向量$\mathbf{A}$的特性,使得我们可对核密度的计算进行加速。其他多项式核函数,例如Epanechnikov核函数也有相同的性质。然而,许多非多项式核函数,例如指数核函数和余弦核函数,无法直接通过这种方式进行分解。

已有的工作尝试使用多项式核函数来逼近非多项式核函数,例如双线性函数逼近 \cite{chan_karl_2019} 和二次函数逼近\cite{chan_quad_2020}。这些方法能够达到较好的精度,然而逼近函数的参数选择极为复杂,并且误差仍然存在且无法收敛。

为了解决这一问题,我们的框架可以直接支持这些非多项式核函数,并且同样可以拆分为查询向量和聚合向量的乘积:
\begin{equation*}
	F_\Gamma(q)= \mathbf{Q}(q) \cdot \mathbf{A}(\Gamma)
\end{equation*}
只需要替换对应的查询向量、聚合向量,并重载对应的计算方式,即可实现不同核函数的调用。


\subsection{指数核函数}

指数核函数的定义如下:
\begin{equation*}
	K(x) = e^{-x}
\end{equation*}

为了简化问题,我们只展示空间维度上的变换,即$K_s(x) = e^{-x}$, $K_t(x) = 1$。此时,核密度为:
\begin{equation*}
\begin{aligned}
	F_{\Gamma}(q) &= \sum_{o_i \in \Gamma} e^{-\frac{d(q, v_c) + d(v_c, p_i)}{b_s}} \\
	&= e^{-d(q, v_c)/b_s} \sum_{o_i \in \Gamma} e^{-d(v_c, p_i)/b_s}
\end{aligned}
\end{equation*}

此时,查询向量和聚合向量分别为:
\begin{equation*}
\begin{aligned}
	\mathbf{Q}(q) &= e^{-d(q, v_c)/b_s} \quad
	\mathbf{A}(\Gamma) &= \sum_{o_i \in \Gamma} e^{-d(v_c, p_i)/b_s}
\end{aligned}
\end{equation*} 

\subsection{余弦核函数}

另一个被广泛使用的核函数是余弦核函数:
\begin{equation*}
	K(x) = \cos(x)
\end{equation*}

余弦核函数的分解需要利用三角函数的变换性质:
\begin{equation*}
	\begin{aligned}
		F_\Gamma(q)
		=& \sum_{o_i \in \Gamma} \cos\left(d(q, v_c)/b_s + d(v_c, p_i)/b_s\right) \\
		=& \sum_{o_i \in \Gamma} \left[\cos \frac{d(q, v_c)}{b_s} \cos \frac{d(v_c, p_i)}{b_s} - \sin \frac{d(q, v_c)}{b_s} \sin \frac{d(v_c, p_i)}{b_s}\right] \\
		=& \cos \frac{d(q, v_c)}{b_s} \sum_{o_i \in \Gamma} \cos \frac{d(v_c, p_i)}{b_s} 
		-  \sin \frac{d(q, v_c)}{b_s} \sum_{o_i \in \Gamma} \sin \frac{d(v_c, p_i)}{b_s}
	\end{aligned}
\end{equation*}

此时,查询向量和聚合向量分别为:
\begin{equation*}
	\mathbf{Q}(q) = 
	\begin{bmatrix}
		\cos \frac{d(q, v_c)}{b_s} \\
		-\sin \frac{d(q, v_c)}{b_s}
	\end{bmatrix}^\top
	\quad
	\mathbf{A}(\Gamma) = 
	\begin{bmatrix}
		\sum_{o_i \in \Gamma} \cos \frac{d(v_c, p_i)}{b_s} \\
		\sum_{o_i \in \Gamma} \sin \frac{d(v_c, p_i)}{b_s}
	\end{bmatrix}
\end{equation*}

\subsection{高维情况下的核密度公式}

无论是空间核函数还是时间核函数,都可以任意选择核函数进行组合:
\begin{equation*}
\begin{aligned}
	F_\Gamma(q) &= \Big[\mathbf{Q}_s(q) \cdot \mathbf{A}_s(\Gamma)\Big] \cdot \Big[\mathbf{Q}_t(q) \cdot \mathbf{A}_t(\Gamma)\Big] \\
	&=\left(\sum_i \mathbf{Q}_i(q) \cdot \mathbf{A}_i(\Gamma)\right) \cdot \left(\sum_j \mathbf{Q}_j(q) \cdot \mathbf{A}_j(\Gamma)\right) \\
	&=\sum_{i,j} \mathbf{Q}_{ij}(q) \cdot \mathbf{A}_{ij}(\Gamma)
\end{aligned}
\end{equation*}

其中$\mathbf{Q}_{ij}(q)=\mathbf{Q}_i(q) \cdot \mathbf{Q}_j(q)$ and $\mathbf{A}_{ij}(\Gamma)=\mathbf{A}_i(\Gamma) \cdot \mathbf{A}_j(\Gamma)$。最后的查询向量和聚合向量的大小$\vert \mathbf{A}_{ij} \vert$ 最大为 $\vert \mathbf{A}_i \vert \cdot \vert \mathbf{A}_j \vert$,即两个维度上的相关变量的两两组合,在实际计算中可以视作常数。


%\subsection{Gaussian Kernel Function?}
%
%Gaussian Kernel Function
%\begin{equation}
%	G(x) = e^{-x^2 \big/ 4\delta}
%\end{equation}
%
%has a squared exponential form and the division equation does not hold. We can of course replace $x$ by $x / 4\delta$ in Taylor's expansion, but we want to find a more efficient and ordinary way to approximate it.
%
%Here we use the sum-of-exponentials(SOE) approximations to expand the squared exponential function into the sum of a series of exponential functions:
%\begin{equation}
%	G(x) \approx \sum_{k=1}^{n} w_k e^{-t_k \vert x \vert \big/ \sqrt{\delta}}
%\end{equation}
%Since we always use a positive distance value, the absolute function can be removed directly. Here, $w_k$ and $t_k$ are two parameters representing different integral region. So the density value can be computed by:
%\begin{equation}
%\begin{aligned}
%	\lambda_{\Gamma_e}(\textbf{q}) = \sum_{i=1}^{\vert \Gamma_e \vert} G(d_i) 
%	= \sum_{i=1}^{\vert \Gamma_e \vert} \sum_{k=1}^{n} w_k e^{-t_k d_i \big/ \sqrt{\delta}} 
%	= \sum_{k=1}^{n} \left( \sum_{i=1}^{\vert \Gamma_e \vert} w_k e^{-t_k d_i \big/ \sqrt{\delta}} \right)
%\end{aligned}\notag
%\end{equation}
%For each $k$, the equation in brackets is a sum of exponential functions, which is discussed in [ch5.1].






%\clearpage